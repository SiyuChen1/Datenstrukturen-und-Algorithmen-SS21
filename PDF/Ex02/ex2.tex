\documentclass[12pt]{scrartcl}
\input{../styles/Packages.tex}
\input{../styles/FormatAndHeader.tex}

\setcounter{sheetnr}{2} % Nummer des Übungsblattes
\setcounter{exnum}{1} % Nummer der Aufgabe

% Beginn des eigentlichen Dokuments

\begin{document}

% Aufgabe 1
\exercise{Verständnis}
\begin{enumerate}
  \item Die folgenden Funktionen in das gemeinsames Koordinatenssystem
  \begin{figure}[!h]
    \centering
      \includegraphics[width=0.5\textwidth]{overview.png}
    \caption{Alle 6 Funktionen in einer Abbildung}
  \end{figure}
  \begin{figure}[!h]
    \centering
      \includegraphics[width=0.5\textwidth]{small_y.png}
    \caption{Nur y-Wert zwischen 0-30 betrachtet wird}
  \end{figure}
  \begin{figure}[!h]
    \centering
      \includegraphics[width=0.5\textwidth]{big_y.png}
    \caption{Größer y-Wert betrachtet wird}
  \end{figure}
  \item $\mathcal{O}(n!) > \mathcal{O}(2^n) > \mathcal{O}(n^3) > \mathcal{O}(n^2log(n)) > \mathcal{O}(n) > \mathcal{O}(log(n))$
  \begin{center}
    \begin{tabular}{ |c|c|c| } 
     \hline
     $f_1(n) = 2^n$ & $\mathcal{O}(2^n)$ & exponenziell \\ 
     $f_2(n) = log(n)$ & $\mathcal{O}(log(n))$ & logarithmisch \\ 
     $f_3(n) = n$ & $\mathcal{O}(n)$ & linear \\ 
     $f_4(n) = n!$ & $\mathcal{O}(n!)$ & faktoriell \\ 
     $f_5(n) = n^3$ & $\mathcal{O}(n^3)$ & polynomiell \\ 
     $f_6(n) = n^2log(n)$ & $\mathcal{O}(n^2log(n))$ & logquadratisch \\ 
     \hline
    \end{tabular}
    \end{center}
\end{enumerate}

% Aufgabe 2
\exercise{Asymptotische Komplexität}
\begin{enumerate}
  \item alg1\\
    for-Schleife i: $\mathcal{O}(n)$\\
    for-Schleife j: $\mathcal{O}(n)$\\
    for-Schleife k: $\mathcal{O}(n)$\\
    Insgesamt: $\mathcal{O}(n)*\mathcal{O}(n)*\mathcal{O}(n) = \mathcal{O}(n^3)$
    \item alg2\\
    while-Schleife: $\mathcal{O}(log{}n)$\\
    for-Schleife i und j: wird nur einmal aufgerufen, $\mathcal{O}(1)$\\
    Insgesamt: $\mathcal{O}(log(n)) + \mathcal{O}(1) = \mathcal{O}(log{n})$
    \item alg3\\
    die erste for-Schleife i: $\mathcal{O}(n)$\\
    die zweite for-Schleife i: $\mathcal{O}(n)$\\
    Insgesamt: $\mathcal{O}(n) + \mathcal{O}(n) = \mathcal{O}(n)$
    \item alg4\\
    $alg4(n) = 2alg4(n-1) = 4alg4(n-2) = ... = 2^{(n-1)}alg2(1)$\\
    Insgesamt: $\mathcal{O}(2^n)$
    \item alg5\\
    for-Schleife i: $\mathcal{O}(n)$\\
    Für alg(n), if-else und for-Schleife werden n mal aufgerufen.\\
    Insgesamt: $n*\mathcal{O}(n) = \mathcal{O}(n^2)$
    \item alg6\\
    for-Schleife i: $\mathcal{O}(n)$\\
    for-Schleife j: $\mathcal{O}(log(n))$\\
    for-Schleife k: $\mathcal{O}(1)$\\
    Insgesamt: $\mathcal{O}(n)*(\mathcal{O}(log(n)) + \mathcal{O}(1)) = \mathcal{O}(nlog(n))$
\end{enumerate}

\end{document}
