\documentclass[12pt]{scrartcl}
% LaTeX Template für Abgaben an der Universität Stuttgart
% Autor: Sandro Speth
% Bei Fragen: Sandro.Speth@iste.uni-stuttgart.de
%-----------------------------------------------------------
% Modul fuer verwendete Pakete.
% Neue Pakete einfach einfuegen mit dem \usepackage Befehl:
% \usepackage[options]{packagename}
\usepackage[utf8]{inputenc}
\usepackage[T1]{fontenc}
\usepackage[ngerman]{babel}
\usepackage{lmodern}
\usepackage{graphicx}
\usepackage[pdftex,hyperref,dvipsnames]{xcolor}
\usepackage{listings}
\usepackage[a4paper,lmargin={2cm},rmargin={2cm},tmargin={3.5cm},bmargin = {2.5cm},headheight = {4cm}]{geometry}
\usepackage{amsmath,amssymb,amstext,amsthm}
\usepackage[lined,algonl,boxed]{algorithm2e}
% alternative zu algorithm2e:
%\usepackage[]{algorithm} %counter mit chapter
%\usepackage{algpseudocode}
% \usepackage{tikz}
\usepackage{hyperref}
\usepackage{url}
\usepackage[inline]{enumitem} % Ermöglicht ändern der enum Item Zahlen
\usepackage[headsepline]{scrlayer-scrpage} 
\pagestyle{scrheadings} 
% LaTeX Template für Abgaben an der Universität Stuttgart
% Autor: Sandro Speth
% Bei Fragen: Sandro.Speth@iste.uni-stuttgart.de
%-----------------------------------------------------------
% Modul beinhaltet Befehl fuer Aufgabennummerierung,
% sowie die Header Informationen.

% Überschreibt enumerate Befehl, sodass 1. Ebene Items mit
\renewcommand{\theenumi}{(\alph{enumi})}
% (a), (b), etc. nummeriert werden.
\renewcommand{\labelenumi}{\text{\theenumi}}

% Counter für das Blatt und die Aufgabennummer.
% Ersetze die Nummer des Übungsblattes und die Nummer der Aufgabe
% den Anforderungen entsprechend.
% Gesetz werden die counter in der hauptdatei, damit siese hier nicht jedes mal verändert werden muss
% Beachte:
% \setcounter{countername}{number}: Legt den Wert des Counters fest
% \stepcounter{countername}: Erhöht den Wert des Counters um 1.
\newcounter{sheetnr}
\newcounter{exnum}

% Befehl für die Aufgabentitel
\newcommand{\exercise}[1]{\section*{Aufgabe \theexnum\stepcounter{exnum}: #1}} % Befehl für Aufgabentitel

% Formatierung der Kopfzeile
% \ohead: Setzt rechten Teil der Kopfzeile mit
% Namen und Matrikelnummern aller Bearbeiter
\ohead{Max Mustermann (1234567)\\
Klaus Kleber (1234568)\\
Melanie Marshmallow (1234569)}
% \chead{} kann mittleren Kopfzeilen Teil sezten
% \ihead: Setzt linken Teil der Kopfzeile mit
% Modulnamen, Semester und Übungsblattnummer
\ihead{Datenstrukturen \& Algorithmen\\
Sommersemester 2020\\
Übungsblatt \thesheetnr}

\setcounter{sheetnr}{1} % Nummer des Übungsblattes
\setcounter{exnum}{1} % Nummer der Aufgabe

% Beginn des eigentlichen Dokuments

\begin{document}

% Aufgabe 1
\exercise{Suchverfahren}
\begin{enumerate}
  \item sequenzielle Suche
    \begin{enumerate}
      \item Folge 1 ohne Optimierung, gesuchtes Element:11\\
        $index = 0, 1 < 11$ \\
        $index = 1, 23 > 11$ \\
        $index = 2, 47 > 11$ \\
        $index = 3, 43 > 11$ \\
        $index = 4, 68 > 11$ \\
        $index = 5, 12 > 11$ \\
        $index = 6, 11 = 11$ return index = 6
      \item Folge 1 mit Optimierung, gesuchtes Element:11\\
        $index = 0, 1 < 11$ \\
        $index = 1, 23 > 11$ return no\_key
      \item Folge 2, gesuchtes Element:48\\
        $index = 0, 12 < 48$ \\
        $index = 1, 13 < 48$ \\
        $index = 2, 23 < 48$ \\
        $index = 3, 47 < 48$ \\
        $index = 4, 48 = 48$  return index = 4 \\
    \end{enumerate}
  \item binäre Suche
    \begin{enumerate}
      \item Folge 1, gesuchtes Element:11\\
        $m = (0 + 14)/2 = 7, F_1[m] = 73 > 11$ \\
        $m = (0 + 6)/2 = 3, F_1[m] = 43 > 11$ \\
        $m = (0 + 2)/2 = 1, F_1[m] = 23 > 11$ \\
        $m = 0, 1 < 11$ return no\_key
      \item Folge 2, gesuchtes Element:48\\
        $m = (0 + 14)/2 = 7, F_2[m] = 73 > 48$ \\
        $m = (0 + 6)/2 = 3, F_2[m] = 47 < 48$ \\
        $m = (4 + 6)/2 = 5, F_2[m] = 68 > 48$ \\
        $m = 4, F_2[m] = 48 = 48$ return index = 4
    \end{enumerate}
\end{enumerate}

% Aufgabe 2
\exercise{verkettete-Listen}
Java-Programmierungsaufgabe

% Aufgabe 3
\exercise{BubbleSort}
\begin{enumerate}
  \item die Liste nach den einzelnen Durchlauf
    \begin{itemize}
      \item 4, 8, 22, 2, 18, 32, 91, 50, 53, 67
      \item 4, 8, 2, 22, 18, 32, 91, 50, 53, 67
      \item 4, 8, 2, 18, 22, 32, 91, 50, 53, 67
      \item 4, 8, 2, 18, 22, 32, 50, 91, 53, 67
      \item 4, 8, 2, 18, 22, 32, 50, 53, 91, 67
      \item 4, 8, 2, 18, 22, 32, 50, 53, 67, 91
      \item 4, 2, 8, 18, 22, 32, 50, 53, 67, 91
      \item 2, 4, 8, 18, 22, 32, 50 ,53, 67, 91
    \end{itemize} 
  \item Für diese einfach verkettete List ist es unmöglich, die Knoten von hinten nach vorne zu iterieren.
\end{enumerate}

% Aufgabe 4
\exercise{Sortierverfahren Komplexität}
\begin{enumerate}
  \item MergeSort\\
    Da die Folge absteigend sortiert (schlechtester Fall) ist, setzen sich MergeSort durch.\\
    Mit O-Notation $O(nlog_2(n))$.
  \item BubbleSort und InsertionSort\\
    Da die Folge F aufsteigend ist, es wird nur $n-1$ Vergleiche gebraucht.\\
    Mit O-Notation $O(n)$.
  \item MergeSort und QuickSort\\
    Da die Folge chaostisch (durchschnittlicher Fall) ist, setzen sich MergeSort und QuickSort durch.
    Da die Aufwandanalyse des MergeSort nlog2(n) ist\\
    Mit O-Notation $O(nlog(n))$.
\end{enumerate}

\end{document}
