\documentclass[12pt]{scrartcl}
\input{../styles/Packages.tex}
\input{../styles/FormatAndHeader.tex}

\setcounter{sheetnr}{1} % Nummer des Übungsblattes
\setcounter{exnum}{1} % Nummer der Aufgabe

% Beginn des eigentlichen Dokuments

\begin{document}

% Aufgabe 1
\exercise{Suchverfahren}
\begin{enumerate}
  \item sequenzielle Suche
    \begin{enumerate}
      \item Folge 1 ohne Optimierung, gesuchtes Element:11\\
        $index = 0, 1 < 11$ \\
        $index = 1, 23 > 11$ \\
        $index = 2, 47 > 11$ \\
        $index = 3, 43 > 11$ \\
        $index = 4, 68 > 11$ \\
        $index = 5, 12 > 11$ \\
        $index = 6, 11 = 11$ return index = 6
      \item Folge 1 mit Optimierung, gesuchtes Element:11\\
        $index = 0, 1 < 11$ \\
        $index = 1, 23 > 11$ return no\_key
      \item Folge 2, gesuchtes Element:48\\
        $index = 0, 12 < 48$ \\
        $index = 1, 13 < 48$ \\
        $index = 2, 23 < 48$ \\
        $index = 3, 47 < 48$ \\
        $index = 4, 48 = 48$  return index = 4 \\
    \end{enumerate}
  \item binäre Suche
    \begin{enumerate}
      \item Folge 1, gesuchtes Element:11\\
        $m = (0 + 14)/2 = 7, F_1[m] = 73 > 11$ \\
        $m = (0 + 6)/2 = 3, F_1[m] = 43 > 11$ \\
        $m = (0 + 2)/2 = 1, F_1[m] = 23 > 11$ \\
        $m = 0, 1 < 11$ return no\_key
      \item Folge 2, gesuchtes Element:48\\
        $m = (0 + 14)/2 = 7, F_2[m] = 73 > 48$ \\
        $m = (0 + 6)/2 = 3, F_2[m] = 47 < 48$ \\
        $m = (4 + 6)/2 = 5, F_2[m] = 68 > 48$ \\
        $m = 4, F_2[m] = 48 = 48$ return index = 4
    \end{enumerate}
\end{enumerate}

% Aufgabe 2
\exercise{verkettete-Listen}
Java-Programmierungsaufgabe

% Aufgabe 3
\exercise{BubbleSort}
\begin{enumerate}
  \item die Liste nach den einzelnen Durchlauf
    \begin{itemize}
      \item 4, 8, 22, 2, 18, 32, 91, 50, 53, 67
      \item 4, 8, 2, 22, 18, 32, 91, 50, 53, 67
      \item 4, 8, 2, 18, 22, 32, 91, 50, 53, 67
      \item 4, 8, 2, 18, 22, 32, 50, 91, 53, 67
      \item 4, 8, 2, 18, 22, 32, 50, 53, 91, 67
      \item 4, 8, 2, 18, 22, 32, 50, 53, 67, 91
      \item 4, 2, 8, 18, 22, 32, 50, 53, 67, 91
      \item 2, 4, 8, 18, 22, 32, 50 ,53, 67, 91
    \end{itemize} 
  \item Für diese einfach verkettete List ist es unmöglich, die Knoten von hinten nach vorne zu iterieren.
\end{enumerate}

% Aufgabe 4
\exercise{Sortierverfahren Komplexität}
\begin{enumerate}
  \item MergeSort\\
    Da die Folge absteigend sortiert (schlechtester Fall) ist, setzen sich MergeSort durch.\\
    Mit O-Notation $O(nlog2(n))$.
  \item BubbleSort und InsertionSort\\
    Da die Folge F aufsteigend ist, es wird nur $n-1$ Vergleiche gebraucht.\\
    Mit O-Notation $O(n)$.
  \item MergeSort\\
    Da die Folge chaostisch (durchschnittlicher Fall) ist, setzen sich MergeSort und QuickSort durch.
    Da die Aufwandanalyse des MergeSort nlog2(n) ist\\
    Mit O-Notation $O(nlog(n))$.
\end{enumerate}

\end{document}
